\newif\ifCodeComment
\CodeCommenttrue     % 代码保留注释
% \CodeCommentfalse  % 代码取消注释

\newif\ifComment
\Commenttrue    % 保留一些自己写的说明,避免未来自己看不懂板子怎么用
% \Commentfalse % 删掉所有的说明

% cpp 代码引用
\newcommand{\Code}[1]{
  \inputminted[
    linenos,
    breaklines,
    fontsize=\small
  ]{c++}{#1}
}

% cpp 代码引用(注释&非注释切换版)
\newcommand{\CodeCommentSwitch}[1]{
  \ifCodeComment
    \Code{#1.cpp}
  \else
    \Code{#1.nocomment.cpp}
  \fi
}

\newcommand{\Comment}[1]{
  \ifComment
    #1
  \else
    % do nothing
  \fi
}

% section 配置,我不想要编号,但是又想进目录,所以这样了
\newcommand{\Sec}[1]{
  \newpage
  \section*{#1}
  \addcontentsline{toc}{section}{#1}
}

% subsection 配置,理由同上
\newcommand{\Subsec}[1]{
  \subsection*{#1}
  \addcontentsline{toc}{subsection}{#1}
}
